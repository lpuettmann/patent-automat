\documentclass[10pt]{beamer}
%\documentclass[14pt, handout]{beamer} % to print out, change to this set-up

\usepackage{amsmath} % to include math
\usepackage{tikz}
\usetikzlibrary{arrows} % to use arrows in tikz

\usepackage{pgfplots}
\usepackage{units}
\usepackage{ulem}
\usepackage{epstopdf} % for making .eps to .pdf
\usepackage{graphicx} % to include graphics
\usepackage{subfigure}
\usepackage{hyperref}
\usepackage{appendixnumberbeamer} 
\usepackage{bbm}
\usepackage[utf8]{inputenc} % this makes it possible to use e

\usepackage{setspace} % to use \setstretch later

\usepackage[australian]{babel} % change date to to European date format

\usetheme{Dresden}

%%% The following section removes the second lighter blue line below the navigation panel %%%
\makeatletter
\beamer@theme@subsectionfalse
\makeatother
%%%%%%%%%%%%%%%%%%%%%%%%%%%%%%%%%%%%%%%%%%%%%%%%%%%%%%%%%%%%%%%%%%%%%%%%%%%%%%%%%%%%%%%%%%%%%%%%%%

\usepackage[flushleft]{threeparttable}  % for nice table

\usepackage{caption,fixltx2e}  % for nice tables
\usepackage[flushleft]{threeparttable}  % for nice tables


%%% I added the following two lines together to edit the color of the navigation bar at the top
\setbeamercolor{section in head/foot}{fg=gray,bg=white} 
%\setbeamertemplate{section in head/foot shaded}[default][10] % I'm not sure what this does


\hypersetup{pdfpagemode=UseNone} % un-comment this if you don't want to show bookmarks open when opening the pdf


%%%%%%%%%% Uncomment the following two lines to use Helvetica as font %%%%%%%%%%%%%%%%%%%%%%%%%%%%
\renewcommand{\familydefault}{\sfdefault} % to use sans-serif font
\usepackage{helvet} % use Helvetica font
%%%%%%%%%%%%%%%%%%%%%%%%%%%%%%%%%%%%%%%%%%%%%%%%%%%%%%%%%%%%%%%%%%%%%%%%%%%%%%%%%%%%%%%%%%%%%%%%%%

% -------- Define new color ------------------------------
\usepackage{color}  

\definecolor{mygrey}{rgb}{0.85,0.85,0.85} % make grey for the table
\definecolor{mytextgrey}{rgb}{0.25, 0.25, 0.25} % make grey for text
% --------------------------------------------------------


%%%%%%%%%%%%%%%%%%%%%%%%%%%%%%%%%%%%%%%%%%%%%%%%%%%%%%%%%%%%%%%
\useinnertheme{circles} % changes the appearance of bullet points for both numbers and circles

\setbeamerfont{page number in head/foot}{size={\scriptsize}} % set size of page numbers

\setbeamertemplate{footline}[frame number]{} % shows the page numbers
\setbeamertemplate{navigation symbols}{} % this turns off the annoying navigation symbols at the bottom of the page

\definecolor{beamer@blendedblue}{rgb}{0.0471,0.298,0.6824} % changed the default color


\setbeamercolor{button}{bg=beamer@blendedblue}

%%%%%%%%%%%%%%%%%%%%%%%%%%%%%%%%%%%%%%%%%%%%%%%%%%%%%%%%%%%%%%%
\title {\textbf{Manual Coding Handbook} \\[1cm]
A New Indicator of Automation \\ Based on Patent Text Search}
\author{
Katja Mann, Lukas P\"uttmann \\  
\textit{University of Bonn}}
\date{\today}




%%%%%%%%%%%%%%%%%%%%%%%%%%%%%%%%%%%%%%%%%%%%%%%%%%%%%%%%%%%%%%%
\begin{document}

\begin{frame}
\titlepage
\end{frame}


\section{Introduction}
\setcounter{subsection}{1} % add this after each \section{...} to start a new section of navigation circles

\begin{frame}\frametitle{Introduction}
	\begin{itemize}	
		\item This document lays out the guidelines for the manual coding of patent texts.
		\item This is used as both a reference and to train and guide human auditors.
	\end{itemize}
\end{frame}


\section{Patents}
\setcounter{subsection}{1} % add this after each \section{...} to start a new section of navigation circles

\begin{frame}\frametitle{What is a patent?}
	\begin{itemize}	
		\item Patents guarantee the intellectual property right to an innovation. In the US this right is for 20 years. 
		\item People or firms apply for patents in a \textit{patent application} in which they describe, why a patent is new, useful and not trivial.
		\item The patent office then looks at the application and decides whether to grant the patent.
		\item If the patent is granted, an official \textit{patent grant text} is written. This is the document we will look at.
	\end{itemize}
\end{frame}


\begin{frame}\frametitle{Patent grant text ...}
... contains most importantly:
	\begin{itemize}	
		\item abstract
		\item detailed description
	\end{itemize}
... and contains also:
	\begin{itemize}	
		\item patent number
		\item name of the inventor
		\item date
		\item citations to other patents
		\item other legal information
		\item drawings
		\item technology classification number
	\end{itemize}
\end{frame}


\begin{frame}\frametitle{Our sample}
	\begin{itemize}	
	\item We study all U.S. utility patents 1976-2015.
	\item We randomly pick an equal number of patents from all years.
	\item We assign each auditor patents. 50\% of patents are assigned to multiple auditors.
	\item The articles are passed in random order (so not in chronological order).
	\end{itemize}
\end{frame}


\section{Automat}
\setcounter{subsection}{1} % add this after each \section{...} to start a new section of navigation circles

\begin{frame}
\textcolor{beamer@blendedblue}{\textbf{Definition:}}

	\begin{itemize}	
	\item An \textit{automat} is a \textbf{device}$^1$ that carries out a \textbf{process}$^2$ \textbf{independently}$^3$.
		\begin{itemize}	
		\item $^1$: could be a a physical machine, computer program, a robot.
		\item $^2$: could be a production process, the act of creation, composing a song, altering something, measuring something, a task
		\item $^3$: without human interference (apart from that needed for supervision, maintenance or at the start)
		\end{itemize}
		\item What we are not looking for: 
			\begin{itemize}
			\item tools, labor efficiency enhancing tools
			\item business or production methods and techniques
			\item new chemicals or medical drugs
			\end{itemize}			
	\end{itemize}
\vspace{0.6cm}
\textcolor{beamer@blendedblue}{\textbf{Our question to you:}}
	\begin{itemize}	
	\item Does the patent we will show you, \textit{even just in a loose sense}, describe an automat?
	\end{itemize}
\end{frame}



\section{Coding}
\setcounter{subsection}{1} % add this after each \section{...} to start a new section of navigation circles

\begin{frame}\frametitle{Practical issues}
	\begin{itemize}	
	\item We will give you an Excel document containing a list of patent numbers such as: 7091471 or 7124853 or 9027105.
	\item Please go to one of the following websites: \\[0.1cm]
 \href{https://www.google.com/patents}{\texttt{https://www.google.com/patents}}\\[0.1cm]
\href{http://patft.uspto.gov/netahtml/PTO/search-bool.html}{\texttt{http://patft.uspto.gov/netahtml/PTO/search-bool.html}}\\[0.1cm]
	and enter the patent number
	\item We recommend using Google Patents first, as it has the nicer interface. If you cannot find it (this happens rarely), please try at the USPTO search interface.
	\item The right patent might not be the first one in the list. In Google Patents, you might find patents from other countries. You identify the right patent as the one starting with ``US". In the USPTO search interface, the right patent is often the last one in the list (don't ask us why...).
	\end{itemize}
\end{frame}

\begin{frame}\frametitle{Practical issues}
	\begin{itemize}	
	\item There are some warning signs, that you might have clicked on the wrong patent: if the main language is not in English (but for example in German or Chinese), then it's probably not an US patent.
	\item If the date of the patent grant text (also called publication date) is before 1976, then something might be wrong. We are only looking at patents that were granted in 1976 and the years after. The application dates can be earlier, though (sometimes a long time before 1976). 
	\item If there is a problem and you cannot figure it out, please indicate that in the comment box.
	\end{itemize}
\end{frame}


\begin{frame}\frametitle{Manual coding}
	\begin{itemize}	
	\item After you have found the patent text, please read through the patent grant text, with a particular emphasis on the abstract and the description and decide: is this an automation patent?
	\item Please use 1 to encode a Yes and 0 to encode a No.
	\item There is also a comment field. Please try to avoid using this, but you can enter here if you encountered technical problems, if something seemed wrong to you or if you encountered some other problem.
	\end{itemize}
\end{frame}


\begin{frame}\frametitle{Sub-classification}
If Yes (the patent is an automation patent) please also decide:
	\begin{itemize}
	\item Is this is a analytical (automation) patent? We mean by that if it does something cognitive and non-physical.
	\item Is this is a manual (automation) patent? So is it a physical machine?
	\end{itemize}
It can also be both. If you think it's neither, that's ok, but please indicate why in the comment box (we haven't encountered that case yet).
\end{frame}


\begin{frame}\frametitle{Excel table after filling it in}
\vspace{-1cm}
\begin{table}
\begin{small}
  \begin{threeparttable}
     \begin{tabular}{ccccl}
        \textbf{Patent number} & \textbf{Classification} & \textbf{Cognitive} & \textbf{Manual} & \textbf{Comment}  \tabularnewline
        \hline
        8851703	& 0  &  &  &	\tabularnewline
        7725116 & 	 &  &  &	Technical problem\tabularnewline
        8975301 & 0	 &  &  &	\tabularnewline
        4716621 & 1	 & 1 &  &	\tabularnewline
        4789151 & 0	 &  &  & 	Not sure\tabularnewline
        4116621 & 1	 & 1 & 1 &	\tabularnewline
        6785151 & 0	 &  &  & 	\tabularnewline
        \hline
     \end{tabular}
  \end{threeparttable}
\end{small}
\end{table}
\vspace{0.5cm}
{\small \textcolor{mytextgrey}{Please note that these patents and numbers have not been checked so don't use them as an orientation for your coding. Also the proportion of patents classified as automation patents is probably not representative.} }

\end{frame}



\begin{frame}\frametitle{When you're done}
	\begin{itemize}	
		\item Please make sure your file is formatted the following way:\\[0.1cm]
\texttt{manclass\textunderscore <FirstnameLastname>\textunderscore <year>-<month>-<day>.xlsx}\\[0.3cm]
		for example, for Lukas on the 11\textsuperscript{th} of May 2015 that would be:\\[0.1cm]
\texttt{manclass\textunderscore LukasPuettmann\textunderscore 2015-05-11.xlsx}\\[0.3cm]
		\item and then send it back to us in an email to\\[0.1cm] \hspace{1cm}\href{mailto:lukas.puettmann@uni-bonn.de}{\texttt{lukas.puettmann@uni-bonn.de}}\\[0.1cm]
	\end{itemize}
\end{frame}



\section{Examples}
\setcounter{subsection}{1} % add this after each \section{...} to start a new section of navigation circles


\begin{frame}\frametitle{Example: automation patent (a clear case)}
\href{https://www.google.de/patents/US5531156?dq=5531156&hl=en&sa=X&ei=KPtRVd3vHYOssgGe2YGQDg&ved=0CCAQ6AEwAA}{\texttt{5531156: Automatic taco machine}}
	\begin{itemize}	
	\item We classify this as an automation patent as it automates the process of cooking tacos, keeping them fresh and filling them with minimal human input or interaction. 
	\item We classify this as not a cognitive, but as a manual automation patent.
	\end{itemize}
\end{frame}


\begin{frame}\frametitle{Example: automation patent (a clear case)}
\href{https://www.google.de/patents/US6362849?dq=6362849&hl=en&sa=X&ei=k-9RVZjJHMmcsgHbvYCIDA&ved=0CCAQ6AEwAA}{\texttt{6362849: Color measuring method and device }}
	\begin{itemize}	
	\item This patent contains a video camera, a processor and the program to measure color. We classify this as an automation patent as it works independently of carrying out the process of recognizing colors.
	\item We classify this as both a cognitive and a manual automation patent.
	\end{itemize}
\end{frame}

\begin{frame}\frametitle{Example: automation patent (a borderline case)}
\href{https://www.google.de/patents/US8497785?dq=8497785&hl=en&sa=X&ei=2O5RVYD3L4GWsAGt0oCgCA&ved=0CCEQ6AEwAA}{\texttt{8497785: Handheld electronic device and method for disambiguation of text input providing suppression of low probability artificial variants}}
	\begin{itemize}	
	\item We classify as an automation patent, as it automates the process of recognizing complete words that are likely to occur from only a few letters that are entered.
	\item What makes this a boarderline case is that it is close to being a tool that is used interactively by people.
	\item As the patent contains both a small machine and the software to it with we classify this as both a cognitive and a manual automation patent
	\end{itemize}
\end{frame}


\begin{frame}\frametitle{Example: automation patent (a borderline case)}
\href{https://www.google.de/patents/US6905407?dq=6905407&hl=en&sa=X&ei=XfJRVYDIAYmosAGKx4CwCw&ved=0CCEQ6AEwAA}{\texttt{6905407: Gaming device having display with interacting multiple rotating members and indicator}}
	\begin{itemize}	
	\item This is an automation patent as it automates the process of dealing hands for ``including but not limited to the games of slot, poker, keno, blackjack, bunco and checkers". What makes this a borderline case is the high degree of human interaction with the machine. Nevertheless, once the machine has been installed, it operates independently and with a high degree of sophistication.
	\item It contains both the physical machine (which also has moving parts) and the software (like programs to play games with) to run on it, so we classify it as both cognitive and manual.
	\end{itemize}
\end{frame}


\begin{frame}\frametitle{Example: \textbf{not} automation patent (a borderline case)}
	\begin{itemize}	
	\item 8457498, Methods and apparatus for target identification
	\end{itemize}
\end{frame}


\begin{frame}\frametitle{Example: \textbf{not} automation patent (a borderline case)}
\href{https://www.google.de/patents/US5505821?dq=5505821&hl=en&sa=X&ei=DvRRVbmqEomrsAHtxIE4&ved=0CCEQ6AEwAA}{\texttt{5505821: Turbulence insert of a papermaking machine}}
	\begin{itemize}	
	\item This is part of a machine which automates the process of paper making. However, it is only a small part of it related to the flow of materials.
	\end{itemize}
\end{frame}


\begin{frame}\frametitle{Example: \textbf{not} automation patent (a clear case)}
\href{https://www.google.de/patents/US5552597?dq=5552597&hl=en&sa=X&ei=mvtRVYXNNomjsAHFi4GYBw&ved=0CCEQ6AEwAA}{\texttt{5552597: Hand-held scanner having adjustable light path }}
	\begin{itemize}	
	\item This is a small hand-held version of a larger desktop scanner which delivers an electronic version of printed text. 
	\end{itemize}
\end{frame}


\begin{frame}\frametitle{Example: \textbf{not} automation patent (a clear case)}
\href{https://www.google.de/patents/US7878521?dq=7878521&hl=en&sa=X&ei=VfZRVe_QMsGzswHZtYHwDA&ved=0CCEQ6AEwAA}{\texttt{7878521: Bicycle frame with device cavity }}
	\begin{itemize}	
	\item This is a bicycle frame that has a cavity for sensors. It does not automate anything.
	\end{itemize}
\end{frame}








\section{FAQs}
\setcounter{subsection}{1} % add this after each \section{...} to start a new section of navigation circles



\begin{frame}\frametitle{FAQs}
	\begin{itemize}	
	\item \textit{What am I looking for?}\\[0.1cm]
	We ask you to classify patents as to whether they are about automation or not. If they are about automation, we also ask you whether they are cognitive or manual.
	\end{itemize}
\end{frame}


\begin{frame}\frametitle{FAQs}
	\begin{itemize}	
	\item \textit{What should I write to say Yes?}\\[0.1cm]
	Please write 1.
	\item \textit{What should I write to say No?}\\[0.1cm]
	Please write 0.
	\item \textit{Can I also leave the field blank if it's a No?}\\[0.1cm]
	We ask you to please fill out the classification with either a 0 or a 1. The 0 is important for us to see that you made a conscious decision about it. If that's somehow not possible, leave it empty and please write this into the comment box. The fields cognitive and manual you can leave empty to signal a No.
		\end{itemize}
\end{frame}


\begin{frame}\frametitle{FAQs}
	\begin{itemize}	
	\item \textit{What if cannot find the patent?}\\[0.1cm]
Sometimes patent don't show up in \href{https://www.google.com/patents}{\texttt{Google Patents}}, but they do appear in the \href{http://patft.uspto.gov/netahtml/PTO/search-bool.html}{\texttt{USPTO}} search interace. In Google Patents you might get several hits and you should search for the patent number starting with US and then the number we give you. On the USPTO website, the patent you're looking for usually shows up as the last hit. If after some searching you cannot find it, please just write something like ``Cannot find" in the comment box.
	\end{itemize}
\end{frame}

\begin{frame}\frametitle{FAQs}
	\begin{itemize}	
	\item \textit{Should I read the whole patent?}\\[0.1cm]
	No, that is likely not necessary (if the text is not very short). We recommend to concentrate on: the title, the abstract (which is a short overview) and the description. The beginning of the description is often most interesting and there is sometimes a section called ``background" which is informative. It also helps to look at the images. It can also help to see which patents were cited (they are likely similar to the one you're looking at) and who applied for it (you might know the firm and infer something from that). You can also use outside knowledge (like Googling a term) if you think that helps. 
	\end{itemize}
\end{frame}	


\begin{frame}\frametitle{FAQs}
	\begin{itemize}	
	\item \textit{Should I enter whether a patent is cognitive or manual even if I don't think the patent is an automation patent?}\\[0.1cm]
	No, that is not necessary, you can just leave the field empty if the patent is not an automation patent.
	\end{itemize}
\end{frame}	


\begin{frame}\frametitle{FAQs}
	\begin{itemize}	
	\item \textit{Do I need any technical knowledge to judge a patent?}\\[0.1cm]
	Patent texts are written in a technical language and they can be very complex. We only ask you to classify the patent as an automation patent if you are \textit{reasonably sure} that the patent is an automation patent. If you absolutely cannot figure it out, then consider writing that in the comment box.
	\end{itemize}
\end{frame}

	
\begin{frame}\frametitle{FAQs}
	\begin{itemize}	
	\item \textit{How often do you expect me to use the comment box?}\\[0.1cm]
	Please only use that box to mark if you encounter any problems or if you are absolutely not sure how to classify the patent.
	\end{itemize}
\end{frame}	

\begin{frame}\frametitle{FAQs}
	\begin{itemize}	
	\item \textit{What if I don't understand what the patent is about?}\\[0.1cm]
	No problem, just judge if you think this patent might have something to contribute to automation. If you're not sure \textit{beyond reasonable doubt} (to use such a legal term), please indicate this in the comment box.
	\end{itemize}
\end{frame}
	
\begin{frame}\frametitle{FAQs}
	\begin{itemize}	
	\item \textit{How long should I take for a patent?}\\[0.1cm]
	We are not quite sure about that yet. If it's less 30 seconds on average per patent, it's probably too fast. If it's more than 5 minutes per patent, it's probably too slow. But the speed will likely differ between people. It's often much faster to see that a patent has \textit{nothing} to do with automation, rather than deciding it is an automation patent. It's ok (and indeed recommended by us) to take more time for those patents that you think might be automation patents.
	\end{itemize}
\end{frame}

\begin{frame}\frametitle{FAQs}
	\begin{itemize}	
	\item \textit{Am I doing you a favor if I classify more patents as Yes?}\\[0.1cm]
	Unfortunately not. We might use your classifications to classify more patents. This will not work if patents are wrongly classified. Also, it makes our work more credible if the manual classifications have a high quality.
	\end{itemize}
\end{frame}
	
\begin{frame}\frametitle{FAQs}
	\begin{itemize}	
	\item \textit{Should I hurry and try to classify as many patents as possible?}\\[0.1cm]
	No, please take your time. It's much more important for us for your coding to accurate than to code more patents. Don't worry if things are going slow, it's ok to classify only few patents. If you're tired and you think you might be making mistakes, then please take a break or stop.
	\end{itemize}
\end{frame}



\section{Thank you.}
\setcounter{subsection}{1} % add this after each \section{...} to start a new section of navigation circles
\begin{frame}
	Last and most importantly: \\[0.2cm]
	\textbf{Thank you for your help}. We appreciate it very much.
\end{frame}


\end{document}
%%%%%%%%%%%%%%%%%%%%%%%%%%%%%%%%%%%%%%%%%%%%%%%%%%%%%%%%%%%%%%%





















