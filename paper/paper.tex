\documentclass[11pt,a4paper,fleqn]{article}

\usepackage{graphicx,hyperref,amsmath,natbib,bm,url}
\usepackage[utf8]{inputenc} % include umlaute

% -------- Define new color ------------------------------
\usepackage{color}  
\definecolor{darkblue}{rgb}{0.15,0,0.37}
\definecolor{darkred}{rgb}{0.35,0,0.08}
% --------------------------------------------------------

\hypersetup{
colorlinks=true,
hypertexnames=false,
colorlinks,		
linkcolor={black},
citecolor={black},
urlcolor={black},
%ocgcolorlinks, % this option changes the link color to black when the document is printed, for legibility in b/w 
pdfstartview={FitV},
unicode,
breaklinks=true
} 
\usepackage{verbatim} %to be able to comment out by using begin{comment} and end{comment}

\usepackage{microtype,todonotes}
\usepackage[australian]{babel} % change date to to European format
\usepackage[a4paper,text={16.5cm,25.2cm},centering]{geometry} % Change page size
\setlength{\parskip}{1.2ex} % show new paragraphs with a space between lines
\setlength{\parindent}{0em} % get rid of indentation for new paragraph
\clubpenalty = 10000 % prevent "orphans" 
\widowpenalty = 10000 % prevent "widows"

\usepackage{mathpazo} % change font to something close to Palatino

\usepackage{longtable, booktabs, tabularx} % for nice tables
\usepackage{caption,fixltx2e}  % for nice tables
\usepackage[flushleft]{threeparttable}  % for nice tables

\newcommand*{\myalign}[2]{\multicolumn{1}{#1}{#2}} % define new command so that we can change alignment by hand in rows

% -------- For use in the bibliography -------------------
\usepackage{multicol}
\usepackage{etoolbox}
\usepackage{relsize}
\setlength{\columnsep}{1cm} % change column separation for multi-columns
\patchcmd{\thebibliography}
  {\list}
  {\begin{multicols}{2}\smaller\list}
  {}
  {}
\appto{\endthebibliography}{\end{multicols}}
% --------------------------------------------------------

\usepackage[onehalfspacing]{setspace} % Line spacing

\usepackage[marginal]{footmisc} % footnotes not indented
\setlength{\footnotemargin}{0.2cm} % set margin for footnotes (so that the number doesn't stick out) 

% -------- Define new line for tables --------------------
\usepackage{array}
\makeatletter  % make line thicker for the table
\newcommand{\thickhline}{%
    \noalign {\ifnum 0=`}\fi \hrule height 1pt
    \futurelet \reserved@a \@xhline
}
\newcolumntype{"}{@{\hskip\tabcolsep\vrule width 1pt\hskip\tabcolsep}}
\makeatother
% --------------------------------------------------------

\usepackage{pdflscape} % for landscape figures

\usepackage[capposition=top]{floatrow} % For notes below figure

%%%%%%%%%%%%%%%%%%%%%%%%%%%%%%%%%%%%%%%%%%%%%%%%%%%%%%%%%%%%%%%%%%%%%%%%%%%%%%%%%%%%%%%%%%%%
\begin{document}


%%%%%%%%%%%%Title %%%%%%%%%%%%%%%%%%%%%
\vspace*{2cm}




\thispagestyle{empty}


\begin{center}
\thispagestyle{empty}
{\huge Labor Demand Effects of Automation:\\[0.2cm] Developing a New Indicator of Automation \\[0.2cm] Based on Patent Text Search}\\[1cm]
\begin{tabular}{l@{\hskip 0.6in}l}
\textbf{{\large Katja Mann}} & \textbf{{\large Lukas Püttmann}}\\
{\small University of Bonn} & {\small University of Bonn}\\
{\small katja.mann@uni-bonn.de} & {\small lukas.puettmann@uni-bonn.de}\\
\end{tabular}\\[0.7cm]
{\large \today}\\[0.7cm]
\end{center}


\vspace*{1.1cm}
\begin{center}
\begin{minipage}{0.65\textwidth}
\textbf{Abstract.} We propose to develop a new industry-level indicator of automation based on searching patent grant texts for keywords. We show the feasibility of such a project by searching the text of all 5.5 mio. patents granted by the U.S. Patent Office from 1976-2015 for words containing the stem ``automat". The 6.9 mio. such matches exhibit a strong and robust upward trend in absolute and relative terms. We are able to match 40 percent of patents up to 2006 with industries through information on the firms being granted a patent. We aim to improve this exploratory first analysis along several dimensions: by refining our search algorithm, by matching patents to industries through existing patent classifications and by potentially extending the coverage back until 1920. We suggest several research avenues to put our new indicator to use: to contribute to the discussion on automation and reshoring and to test if sector-level technological progress in automation has led to changes in employment in such sectors.
\end{minipage}
\end{center}

%\newpage
%\tableofcontents 


\newpage
\section{Introduction}
The rapid technological change of the recent decades has fundamentally altered production processes: it is becoming more widespread to use machines for tasks previously executed by humans. This process has long since attracted the interest of economists. In labor economics, researchers try to assess the effects of automation on labor demand, wages and employment. 

In order to carry out meaningful economic analyses on those topics, good measure of automation are needed. So far, research has often equated automation with routinization, arguing that computers and machines are good at carrying out tasks that require little variation, while they fail at complex analytic or manual tasks. A widespread index of automation, proposed by \cite{ALM2003} and further refined by various authors, e.g. \cite{AKK2008} and \cite{AD2013}, uses data on occupational titles (e.g. by the U.S. Occupational Information Network (O*NET) database). However, using the routine task content of a specific occupation as a measure for automation is problematic for several reasons: first, it is purely static, and builds on present-day knowledge on which tasks are automizable. Routine-task indices enable researchers to assess which tasks can be replaced by machines given the level of technological progress that we observe today, but they do not allow for predictions about future automation and its likely effects. Second, there is no way of knowing whether those occupations that experts consider routinizable are indeed prone to be replaced by machines. Whether or not automation is actually carried out depends on many things other than pure technical feasibility, e.g. on cost considerations of firms or on research and development efforts. 

We propose to develop a new industry-level indicator of automation based on searching patent grant texts for keywords, which promises to remedy several of these shortfalls. Patents are an objective and transparent measure of technological progress and records on U.S. patent grant texts exist since 1790. This could potentially allow us to test how changes in the technology frontier on automation in an industry affect jobs in that industry. 

We illustrate the feasibility of such a project by searching the text of all 5.5 mio. patents granted by the U.S. Patent Office from 1976-2015 for words containing the stem ``automat". The 6.9 mio. such matches exhibit a strong and robust upward trend. We are currently able to match patents with industries drawing on work by \cite{HJT2001} who match patents with the firms that are granted the patents. We aim to improve our matching by using the existing highly detailed technological classifications which link patents to around 400 technical classes and roughly 150'000 subclasses. 


To put our new indicator to use in the field of labor economics, we suggest several research avenues: to investigate the effects of sector-level technological progress in automation on employment, and, complementing the automation indicator with indices of offshoring, to contribute to the discussion on reshoring, the relocation of previously offshored production back to the home country.

In what follows, we first present some background information on the use of patent grant texts as an indicator of technological progress, before describing how we construct our automation index based on this data source. We then explain how we assign patents to different industries. In the last part, we propose different research questions that could be answered with the help of the new indicator, and show how these projects could be carried out.


\section{Patent Grant Texts as an Indicator of Technological Progress}
Patent grant texts are a rich publicly-available data source that allows for a measure of technological progress. Patent grant texts are highly standardized and applicants have incentives to provide exact and correct information about their innovations. In return for disclosing the content of the innovation to the public, a temporary intellectual property right is granted. 

The patent grant document includes the patent number, the name of the inventor, date, the patent classification, citations to other patents, an abstract, (most importantly) a description and a range of other legal information and drawings, all of which data that can be searched and analyzed. 

We are not the first to analyze patent data: \cite{HJT2001} provide data on patent citations and link those to publicly-traded firms on Compustat. A number of papers, such as \cite{AAC2014} and \cite{ABP2015} have made use of patent citations. However, contrary to those papers, we concentrate on the patent's \textit{content}, not on its relation to other patents. As \citep[p.~290]{MVS2010} note, 
\begin{quotation} 
``[...] the majority of studies and analyses undertaken in [patent and publication text search] focus on mere counts and/or quantitative attributes of relevant documents (e.g. number of publications, patents, number of references, number of citations); less attention is paid to analysis in which the content of the underlying documents is the focal point of attention."
\end{quotation}

Text mining and search is starting to be more widely used in empirical social science. \cite{BBD2013} search newspaper articles to measure economic policy uncertainty. \cite{GS2010} analyze search newspapers  for phrases to measure political slant. Advantages of patent texts over newspaper articles are the greater standardization, the incentive to deliver correct information, the precise technical nature allowing for measures of technology and easy public access to these documents. \cite{HMP2014} apply measures of similarities of speak developed in computational linguistics to the records of decision making process of central banks.

In our first explorative analysis we concentrate on the number of appearances of the phrase ``autom", which brings matches such as ``automatic", ``automated", ``automatically" and ``automation". We interpret this as a proxy for the contribution of a patent of enabling the automation of tasks. We admit that this measure is crude and aim to improve our search algorithm by including other phrases such as ``mechanic".

Another way to improve the quality of our matches could be to check for the text surrounding the match. If the surrounding text passages of a patent's matches are all highly alike, the number of matches might not carry much information on the how large the patent's contribution to automation is. As an example of this, look at patent  4649323 ``Microcomputer-controlled light switch" from 1987. Here ``automat" appears 26 times, but the term ``fade[r]" appears in the vicinity (next 2 words) of all of the matches. If the phrase, however, appears in a more diverse surrounding, this could be stronger indication of the patent's content being about automation. An example here would be patent 4648731 ``Error correction member positioning system for a printer" from 1987. In its patent grant text ``automat" appears 17 times, in all cases as ``automatically". The phrase appears in diverse settings such as ``automatically [forwarding, backspacing,  separating, backing, moved, returned]".

We could also take the origin of the applicant into account. Foreign applicants face significantly higher costs, so their patent might be assumed to be of higher value \citep{HS2005}.


\section{Construction of the Indicator}
\subsection{Data source}

The United States Patent and Trademark Office (USPTO) grants and administers patent applications for the United States. Patent grant texts are publicly available and searchable.\footnote{\href{http://patft.uspto.gov/netahtml/PTO/search-bool.html}{The USPTO provides a search interface for this at: \texttt{patft.uspto.gov/netahtml/PTO/search-bool.html}}} Google provides bulk downloads of patent information free of charge.\footnote{\href{www.google.com/googlebooks/uspto-patents.html}{\texttt{google.com/googlebooks/uspto-patents.html}}} The data include US, European and global patents. 

Electronic records on patent grant texts exists since 1976. Grants from 1920 onwards have also been transferred to electronic format by optical character recognition. However, the recognition technology makes frequent mistakes, which could lead to wrong classifications. More likely it would lead to far fewer matches. In future work, these files might be included through an algorithm that corrects for mistakes.

There are several changes in the formatting of the files. Until 2001, the files are in ASCII plain text and thereafter in eXtensible Markup Language (XML). There is another formatting change after 2004.


\subsection{Implementation of search algorithm}
We obtain the full grant texts for all US patents of 1976-2015.\footnote{\href{http://www.google.com/googlebooks/uspto-patents-grants-text.html}{\texttt{google.com/googlebooks/uspto-patents-grants-text.html}}}  Every file contains the patents of one week starting Tuesday. The total size of the decompressed files is about 300 gigabytes.\footnote{\cite{F2013} provides some advice how to handle these files.}  

We first construct a patent index by searching containing of each number's unique patent number, its position of appearance in the file and its technological classification. We then do a linear case-insensitive regular expression search for the stem ``automat'' for every patent grant text and record the number of matches. We disregard the order in which word in the corpus appear, but focus only on the word counts. This helps to identify \textit{what} a text covers, but now \textit{how}. Stemming our words of interest such as ``automation", ``automated", ``automatic" to ``automat" is a crude, but effective way of searching a large corpus of documents \citep{MRS2009}.

We do plausability checks of our matches by looking at the context in which matches appear. The scale of the search only allows for inspecting a relatively small number of matches. We do this once by searching by matches, e.g. match number 100, 101, and so on. We further randomly draw matches and look at the context in which the matches appear. This first superficial test shows that patents which include the expression ``automat'' provide a good indication of whether a patent contributes to the technological level of automation in a sector. Following \cite{BBD2013}, who use human auditors to check their results, in future analyses we could formalize such testing to ensure a high quality of matches. We exclude the name of the patent holder from our search corpus to eliminate a possible source of errors with names such as ``Automated Truck Systems Inc.''.\footnote{Actually, we have only been able to make this correction for patents until 2001. This biases our number of matches (very slightly) upwards after 2001. We plan to correct this.} 

\subsection{Results}
In total, we identify around 5.5 mio. patents. Figure~\ref{fig:keyword_match_over_time} displays the number of weekly matches of the keyword ``automat" from 1976 to 2015. Over the whole period, we find around 6.9 mio. matches of ``automat'' in the patent grant texts. We can observe a clear upward trend from around 1500 matches in the 1970s and 1980s, 4000 around 2001 up to an average of 10100 matches by 2014.


\begin{figure}[tb]
\caption{Number of weekly matches of word stem ``automat" in US patents, 1976-2015}
	\centering
	\includegraphics[trim=1cm 0cm 0cm 0cm, clip=true, totalheight=0.38\textheight]{./figures/keyword_matches_weekly_1976-2015.pdf}
	\floatfoot{\textit{Note:} Every data point displays sum of matches from searching through one file of weekly patent grant texts; total identified patents: 5.5 mio.; total number of keyword matches: 6.9 mio.\\[0.1cm]
	\textit{Source:} USPTO, Google and own calculations.}
	\label{fig:keyword_match_over_time}
\end{figure}



Figure~\ref{fig:summary_stats} shows a summary of the matches. The increase in keyword matches could be due to the general increase in the number of patents (see panel A) from around 70'000 to 320'000 patents. 


\begin{figure}[tb]
\caption{Number of identified patents, 1976-2014}
	\includegraphics[trim=0cm 0cm 1cm 0.5cm, clip=true, totalheight=0.25\textheight]{./figures/barchart_identified_patents_1976-2014.pdf}
	\label{fig:summary_stats}
	\floatfoot{\textit{Note:} Identified patents are those found through the search algorithm; the bottom darker-shade bars are the number of distinct patents with at least one match \\[0.1cm]
	\textit{Source:} USPTO, Google and own calculations.}
\end{figure}

However, the mean number of matches has also shifted upwards from around 0.93 to 1.6 (panel C). \footnote{The median is reliably at zero.} Panel D shows that this trend is robust when we exclude the top percentile of matches, which contains outliers of patents that have hundreds of matches. Figure~\ref{fig:match_per_nrpatents} plot the weekly average over average of the number of keyword matches per week which is upwards trending. 


\begin{figure}[tb]
\caption{Weekly averages of number of average matches per patent, 1976-2015}
	\centering
	\includegraphics[trim=1cm 0cm 1cm 0cm, clip=true, totalheight=0.4\textheight]{./figures/match_over_nrpatents_weekly_1976-2015.pdf}
	\floatfoot{\textit{Note:} Every data point displays sum of matches from searching through one file of weekly patent grant texts divided by the number of identified patents for that week; the solid line is the HP-filtered trend with $\lambda = 100'000$; the size of markers corresponds to number of patents in week.\\[0.1cm]
	\textit{Source:} USPTO, Google and own calculations.}
	\label{fig:match_per_nrpatents}
\end{figure}



\subsection{Developing an index for automation}
We propose to develop a new sector-level index of automation based on the keyword search. The index should rise in the number of matches, but we want the marginal benefit of an additional match to decrease.  A possible candidate for such an index is $x_i = log(1+m_i)$, where $m_i$ are the number of keyword matches per patent and $x_i$ is the index assigned to patent $i$. 

As a further option we could weigh this index by the number of times a patent has been cited by other patents using the data provided by \cite{HJT2001}, however correcting for the fact that older patents have been cited more often. Such a citation-weighted index could take the form $x_i = log(1+m_i) [1+log(1+\frac{c_i}{C_t})]$, where $c_i$ are the patent $i$'s citations over the whole period and $C_t$ are the total number of citations of patents in the same year over the whole period. A potential cost of including patent citations would be limiting the number of patents to the roughly 3 mio. patents and the time up to 2006 that \cite{HJT2001} (and later revisions) cover. 


\section{Linking Patent Citations to Industries}
In order to use our patent data for economic analyses other than at the aggregate level, we need to match the patent classes with the industries where these patents are applied. In general, this is not an easy task since the same patent might be of use in various different industries. The industry to which the patent assignee belongs might not be informative in this context because many large companies - which are, in general, those companies most likely to innovate - are active in multiple different fields. Having warned about these possible shortcomings, we still believe that patent-industry matching yields meaningful insights.%\linebreak

There are two ways of doing this: The first is to use the patent's technical class to assign a patent to an industry. The USPTO provides a convergence table that links patent classes with NAICS industries, which has been compiled manually by trained analysts.\footnote{\href{http://www.uspto.gov/web/offices/ac/ido/oeip/taf/data/naics_conc/2013/read_me.txt}{\texttt{http://www.uspto.gov/web/offices/ac/ido/oeip/taf/data/naics\_conc}}} However, it only contains manufacturing industries, and their number is restricted to 26. 

The alternative, which we choose for this first analysis, is to make use of the database developed by \cite{HJT2001}. For the years 1976-2006, the authors have matched about 70 percent of all patents granted by patent numbers with assignee names and the corresponding corporations in Compustat, which comprises all U.S. firms that are traded on the stock market. Using this information, we were able to match the Compustat identification number with the number of patent matches which we derived by our algorithm. We used information available on the Compustat website to match companies to NAICS industries. 

\begin{figure}[tb]
	\caption{Automation index at first-digit NAICS, 1976-2006}
	\centering
	\includegraphics[trim = 0cm 0.4cm 0cm 0cm, clip=true, totalheight=0.45\textheight]{./figures/automation_naics_firstdigit.pdf}\\
	\floatfoot{\textit{Source}: USPTO, Google and own calculations.}
	\label{fig:automation_naics_firstdigit}
\end{figure}


Figure~\ref{fig:automation_naics_firstdigit} shows our automation index disaggregated for first-digit NAICS industries. As can be seen, the highest increase in automation was in the manufacturing and the IT, finance and real estate sector. 




%Table \ref{table:industry_linked2matches2001} shows some examples of matched industries for 2001. This procedure can easily be extended to all years until 2006. In 2001, our automation index is defined for 470 NAICS industries. In 141 of these, no patents with keyword matches were granted.

\begin{comment}
\begin{table}
\begin{small}
  \begin{threeparttable}
    \caption{{\normalsize Examples of matched industries with patents, 2001}}
    \label{table:industry_linked2matches2001}
     \begin{tabular}{llr}
        \toprule
        \textbf{NAICS} & \textbf{Industry name} & \textbf{Number matches}  \tabularnewline
        \midrule
        334413	& Semiconductor and Related Device Manufacturing  & 6'867							\tabularnewline
        511219 	& Other Computer Related Services$^1$				& 6'418									\tabularnewline
        541512 	& Computer Systems Design Services				& 5'692									\tabularnewline
        \myalign{c}{\vdots}	& \myalign{c}{\vdots}			& 		\myalign{r}{\vdots}							\tabularnewline
        3335 	& Motor Vehicle Parts Manufacturing				& 2									\tabularnewline
        3363 	& Metalworking Machinery Manufacturing				& 1									\tabularnewline
        \myalign{c}{\vdots}	& \myalign{c}{\vdots}			& 		\myalign{r}{\vdots}							\tabularnewline
        3112 	& Grain and Oilseed Milling				& 0									\tabularnewline
        \myalign{c}{\vdots}	& \myalign{c}{\vdots}			& 		\myalign{r}{\vdots}							\tabularnewline
        \midrule
        	& Mean matches per industry  & 200.1							\tabularnewline
        	& Total matches linked to industries  & 94'046							\tabularnewline
        	& Total number of matches  & 231'514							\tabularnewline
        	& Share of matches linked to industries  & 41\%							\tabularnewline
        	& Total number of patents identified  & 184'172							\tabularnewline
        \bottomrule
     \end{tabular}
    \begin{tablenotes}
      \footnotesize
      \item $^1$includes software installation services and computer disaster recovery services 
      \item 	\textit{Source:} USPTO, Google and own calculations.
          \end{tablenotes}
  \end{threeparttable}
\end{small}
\end{table}
\end{comment}

\section{Comparison Automation Index - Routine Task Index}

We have described above how our automation index differs from the automation index that is most widely used in the literature, the Routine Task Index (RTI). We are interested in how these indices correlate. This can help us assess the quality of our index, as well as the new value it provides.\\
In order to compare the indices, it is necessary to aggregate the occupation-level RTI on the industry level. This task requires not only that we know which occupations are used in which industry; we also need to be able to weigh them according to their share in labor input in each industry.

To assign occupations to industries, we can make use of the surveys carried out by the U.S. Census Bureau. In the American Community Survey (ACS), the Current Population Survey (CPS), and the Survey of Income and Program Participation (SIPP), to name just the largest ones, survey participants provide information on their occupation as well as on the industrial sector where they work. The Census Bureau has processed this information to generate correspondence tables. The industrial classification used by the Census Bureau is the Census Industry Codes (CIC). These can be matched with NAICS industries using specific industry codes crosswalks, also provided by the Census Bureau.\footnote{Note that these are not one-to-one; often, a CIC is assigned various NAICS or the other way round. Also, CICs differ slightly across surveys, which complicates the assignment to NAICS industries. In general, it is not possible to in this way obtain reliable information on more than the first two NAICS digits.}

\cite{ALM2003} weigh occupations by their respective shares in employment in a CIC industry. Their dataset contains the necessary components to construct the RTI and an assignment of occupations to industries weighted by their 1991 share of employment. We use their quantification of the routine, analytical and manual task components in O*NET data to construct the RTI. We apply the \cite{AD2013} definition of the RTI, which is more widely used than the original one by \cite{ALM2003}. According to \cite{AD2013},
\begin{equation}
 \text{RTI} = ln(\text{routine}) - ln(\text{manual}) - ln(\text{analytical})
\end{equation}  

Since \cite{ALM2003} use their data to also carry out analyses for different educational groups, this allows for the creation of separate RTIs for four different educational groups. While our main interest is in comparing our automation index with the aggregate RTI, this additional information is nevertheless helpful in assessing the quality of our index. 

Applying the procedure described above, we are able to match the RTI with our automation index at the level of the first two digits of the NAICS classification codes. There are altogether 24 two digit-NAICS industries for which we have observations.

\begin{comment}
\begin{table}[htbp]\centering \small \caption{Cross-correlation table one-digit NAICS\label{corrtable1}}
	\begin{tabular}{l  c  c  c  c  c  c }\hline\hline
		\multicolumn{1}{c}{Variables} &automation\_index&RTI\_all&
		RTI\_HSD&RTI\_HSG&RTI\_SMC&RTI\_CLG\\ \hline
		automation\_index&1.000\\
		RTI\_all&0.607&1.000\\
		RTI\_HSD&0.764&0.878&1.000\\
		RTI\_HSG&0.693&0.935&0.870&1.000\\
		RTI\_SMC&0.541&0.940&0.724&0.921&1.000\\
		RTI\_CLG&-0.418&0.191&-0.212&0.216&0.421&1.000\\
		\hline \hline 
	\end{tabular}
\end{table}
\end{comment}

\vspace{0.5cm}
\begin{table}[htbp]\centering \small \caption{Cross-correlation table two-digit NAICS\label{corrtable2}}
	\begin{tabular}{l  c  c  c  c  c  c }\toprule
		\multicolumn{1}{c}{Variables} &automation\_index&RTI\_all&
		RTI\_HSD&RTI\_HSG&RTI\_SMC&RTI\_CLG\\ \midrule
		automation\_index&1.000\\
		RTI\_all&0.145&1.000\\
		RTI\_HSD&0.364&0.871&1.000\\
		RTI\_HSG&0.414&0.820&0.823&1.000\\
		RTI\_SMC&0.180&0.903&0.715&0.904&1.000\\
		RTI\_CLG&-0.570&0.344&0.003&0.207&0.543&1.000\\
		\bottomrule 
	\end{tabular}
\end{table}
\vspace{0.5cm}

As can be seen, the overall correlation of the automation index with RTI is small but positive, at 0.145. In order to interpret this value, it is insightful to consider the correlations of our index with RTI for different educational groups: Whereas it is largely positive for high school dropouts (HSD) and high school graduates (HSG), it is much smaller for those jobholders that have some college eduation (SMC) and largely negative for those with completed college education (CLG). Thus, it can be said that automation and routinizability describe similar things in lower educational groups, while they diverge for occupations that require a lot of education.

One possible explanation for this is that the type of automation captured by patents is mostly meant to replace low-education routine jobs, for example a patent of a robot working in an assembly line. However, the type of automation technology replacing highly educated employees is likely to be less tangible: it could, for example, be a computer software that facilitates the filing of an electronic tax declaration and thus replaces tax accountants. Note that this software, if patented at all, would also probably be filed in a different sector: not by an accountancy firm but by an IT firm.



\section{Labor Demand Effects of Automation}

\subsection{Direct employment effects}
It is natural to assume that the tremendous increase in patents on automation over the last decades will have major effects on the labor market. While on the one hand, labor productivity in the U.S. economy has been rising continuously, on the other hand, the employment-to-population ratio is lower today than any time in the last 20 years, and the real income of the median worker is lower than in the 1990s \citep[p.~164]{BM2014}. 

A growing number of labor economists has focused on the link between automation and employment, see for example \cite{ALM2003}, \cite{GM2007} and \cite{AKK2008}. Mostly on the occupation level, these papers aim to find out whether the technological change induced by new machines is skill-biased (in that it benefits high-skilled workers and harms low-skilled workers) or whether occupations that require a higher degree of routine tasks are more in danger of being replaced by machines than those that comprise non-routine analytic or manual tasks. There are two main reasons why, despite the large number of publications on the subject, we expect our approach to be a novel contribution: first, constructing an indicator without prior restrictions on what we expect automation to be like in the future; second, building an industry-level instead of an occupation-level indicator.

\subsubsection*{Avoiding imposing prior expectations on automation index}
As argued above, existing automation indices often use prior knowledge to assess what we expect machines to be able to do in the future. In particular, the manual part of tasks is often considered to not be automizable. \cite{BM2014} argue that recent advances in robotics suggest otherwise. 

\cite{AKK2008} and \cite{AD2013}, building on \cite{ALM2003} construct an automation index as the log of routine task importance divided by the manual task importance. Similarly, \cite{GMS2014} reduce their indicator for those tasks that includes abstract and service task importances. Our approach does not penalize mechanical work aspects, but rather builds an indicator at a deeper level interpreting patents as indicative of technological possibilities.


\subsubsection*{Towards an industry-level indicator of automation}
Second, we construct an industry-level indicator while existing classifications usually work at the occupation level. This offers the possibility to approach the academic debate on automation from a different angle. Having matched our patent counts with NAICS industries, we obtain an industry-time panel dataset. Employment data is also available at the industry level: the U.S. Bureau of Labor statistics (BLS) publishes data for NAICS industries on a monthly basis, which is complete from 1990 onwards, but even goes back as far as the 1930s for some industries. Thus, our dataset will comprise at least 15 years of monthly employment and automation data for around 500 industries. It will allow for a detailed analysis of which industries gain and which lose from automation in terms of employment. The effect of automation on employment is most likely going to be lagged, since the granting of a patent will not immediately result in its widespread application. We could approach the identification of the effect for each industry through time series techniques such as cointegration or VAR analysis. Due to the lagged response of employment, we can use present-day innovations to make predictions about future employment. This could be of strong interest to both researchers and policy-makers.


\subsection{Indirect employment effects through reshoring}
Next to automation, the second major development in labor demand over the last decades has been the relocation of production sites to foreign, mostly less developed countries. ``Offshoring", which is defined by the Palgrave Dictionary of Economics as the ``import of parts or services from related overseas suppliers, such as foreign subsidiaries", has become the research focus of trade as well as labor economists. Studies that analyze the effects of offshoring on employment include those by \cite{FH1999}, \cite{GR2008}, \cite{EHMP2014} and \cite{HJMX2014}), but only few have considered offshoring jointly with automation. A noteworthy exception is \cite{GMS2014}, who estimate which of the two developments has a larger effect on employment in the United States. To the best of our knowledge, no research paper has so far investigated whether automation also affects home-country employment indirectly through offshoring.\footnote{This link is suggested, but not elaborated on, by \cite{BM2014}.} It might be the case that the introduction of automation techniques in the production process affects the offshoring behavior of firms: When machines decrease unit labor costs at home, it might become cheaper for a firm to relocate production back from the foreign country to the home country.\footnote{A necessary condition is that it is cheaper for firms to set up machines where the final market is than in the offshore location. \cite{BMS2014} argue that there will be added value from shortening delivery times and reducing inventory costs, which will offset the cost advantage from having to pay machine operators in the home country higher wages than abroad.}
This process, called ``reshoring", might bring about positive employment effects in the home country.

\subsubsection*{Measuring offshoring}
We have shown above that in order to quantify the effects of automation, it is key to have an appropriate measure at hand; the same argument holds for offshoring. In the literature, different indices have been proposed to measure offshoring.

\cite{B2009} creates an index of offshorability based on information contained in O*NET. By means of a classification tree, occupations are sorted into four different categories that range from ``Non-offshorable" to ``Highly offshorable". He presents a ranking of 291 occupations according to their offshorability, but does not provide a cardinal scale. The same methodology is applied to Germany by \cite{SL2009}, who use the BERUFENET database of the Federal Employment Agency (Bundesagentur für Arbeit).

\cite{BK2013} use worker-level survey data gathered as part of the Princeton Data Improvement Initiative (PDII) to create more nuanced indices of offshorability. The approx. 2,500 survey participants in the U.S. were asked to classify their jobs according to how easily they could be performed from a distant location, and to what extent face-to-face interaction was required. Using this information, \cite{BK2013} create an index of offshorability that distinguishes five different categories.This index is for the occupation level, but the PDII dataset allows for aggregation into NAICS industries as well. 

Note that both indices do not grasp the realised offshoring of jobs, but the potential of the jobs still performed at home to be offshored. Neither are those occupations that have already been offshored captured by the indices, nor do they contain any information on how likely it is that these jobs are effectively going to be offshored, or when. In parallel to the argument that was made above for automation indices, such a type of offshorability index describes a potential development from the viewpoint of today's technological advances; it does not describe a certain future.

An alternative index has been proposed by \cite{GMS2014}, who analyze data from the European Restructuring Monitor (part of the Eurofound), which provides summaries of news reports on offshoring plans of European companies. They sum up all events by occupation type of the offshored jobs and rescale to obtain a measure with a distribution of zero mean and unit standard deviation. However, it seems questionable how complete and how reliable the data is.\footnote{Yet another alternative would be to use data on foreign direct investment, but the problem is that there is no means of knowing whether the goods produced in the foreign production site are sold in the home market (which would be required for offshoring) or in the foreign market.}

Given these shortcomings of existing indices, we could try to build a new offshoring through patent text search. The keywords to look for are, however, less clear than in the case of automation. One possibility would be a combination of words that describe geography and production, e.g. the stems ``global", ``international", ``cross-border", ``long-distance", ``foreign" combined with ``value chain", ``investment", ``production site", ``subsidiary", ``transaction costs", ``shipment" or ``intermediate inputs". These are preliminary thoughts only; developing an informative index of offshoring based on patent data is one of the goals of our research paper. Note that this database, in addition to applying a more useful definition of offshoring, also offers the advantage of analyzing changes over time since it could be calculated yearly, monthly or weekly.

\subsubsection*{Offshoring and automation}

In absense of an appropriate measure of offshoring, any analysis of the link between offshoring and automation is necessarily preliminary and incomplete. However, for illustrative purposes, we plot the offshorability measure of \cite{BK2013} against our measure of automation in figure~\ref{fig:offshoring_vs_automation}: on the x-axis, the offshorability index is aggregated at the industry level, ranking from -2 (``not offshorable") to 2 (``easily offshorable"). On the y-axis is the industry sum of automation matches in the patent database for 2001.\footnote{The PDII survey was carried out in 2008, so ideally, we would use patent data of 2008. Since our patent dataset so far reaches only until 2001, we chose the last year available as the one closest to the survey year.} The correlation is positive and significant, but rather small at 0.02.

\begin{figure}
	\centering
	\caption{Automation vs. Offshoring}
	\label{fig:offshoring_vs_automation}
	\includegraphics[totalheight=0.35\textheight]{./figures/automation_vs_offshore.pdf}
	\floatfoot{\textit{Source:} \cite{BK2013}, USPTO, Google and own calculations.}
\end{figure}

It can be seen that there are certain types of industries where a lot of automation has taken place in 2001, but which are not offshorable at all, like ``Support Activities for Oil and Gas Operations". Others, like ``Medical Instrument Manufacturing" are both offshorable and automizable. It will be interesting to think about the potential for reshoring across these different types of industries. Medical Instrument Manufacturing could be a candidate for reshoring if in the production process of these instruments, foreign workers are being replaced by machines once innovations have taken place. But it could as well be that Medical Instrument Manufacturing will never be offshored because the use of machines has decreased production costs in the home country already at an earlier time.

This short discussion makes clear that the timing of offshoring and automation is crucial: Only if offshoring happens first will automation potentially lead to reshoring. Once we have developed a new offshoring index based on patent data, we will be able to detect time trends in the correlation, on the aggregate as well as on the industry level. If the correlation has increased over time, this might be an indicator that foreign labor and machines are becoming ever closer substitutes. It will also be interesting to see for which industries correlation is strong and has increased and for which it is weak and has decreased.

\subsubsection*{Developing model of reshoring}

In order to establish a causal link between automation and offshoring and hence to provide support for the reshoring hypothesis, a theoretical model is needed. We plan to develop such a model, using as a starting point some existing models of offshoring (e.g. \cite{GR2008}) and automation (e.g. \cite{ALM2003}, \cite{L2011}). In this model, complementarity and substitutability of home labor, foreign labor and machines will play a major role. One difficulty will be to transfer the mechanics of the existing occupation-level models to the industry level. The model could be tested empirically once we have the necessary data at hand. 

\subsection{Other possible uses of database not linked to employment}
Although not of immediate interest to us in the context of this research project, there are many other possible applications of the automation database, which could be the subject of future research projects: (1) automation and sectoral productivity: does technological change increase (labor) productivity more in some sectors than in others? (2) comparing U.S. automation patents to European automation patents (using additional data from the European Patent Office or other sources): where is automation higher? Are there sector-level differences and do these link to comparative advantage and/or institutional characteristics? (The political economy literature on varieties of capitalism suggests that U.S. firms should innovate more in IT, pharmaceutics, biotechnology and related sectors whereas continental European firms should innovate more in manufacturing.)


%%%%%%%%%%%Bibliography%%%%%%%%%%%
%\newpage
\vspace*{0.5cm}
\setstretch{1} % Set line spacing to 1 for the bibliography.
\begin{thebibliography}{9}
	
	\bibitem[Acemoglu, Akcigit and Celik(2014)]{AAC2014} \textbf{Acemoglu, Daron, Ufuk Akcigit and Murat Alp Celik} (2014). ``Young, Restless and Creative: Openness to Disruption and Creative Innovations." NBER Working Paper No. 19894

	\bibitem[Arora, Belenzon and Patacconi(2015)]{ABP2015} \textbf{Arora, Ashish, Sharon Belenzon and Andrea Patacconi} (2015). ``Killing the Golden Goose? The Decline of Science in Corporate R\&D." \textit{NBER Working Paper No. 20902}.
	
	\bibitem[Autor and Dorn(2013)]{AD2013} \textbf{Autor, David H. and David Dorn} (2013). ``The Growth of Low-Skill Service Jobs and the Polarizatio of the US Labor Market." \textit{American Economic Review}, 103(5): 1553-1597
	
	\bibitem[Autor, Katz and Kearney(2008)]{AKK2008} \textbf{Autor, David H., Lawrence F. Katz and Melissa S. Kearney} (2008). ``Trends in US wage inequality: Revising the revisionists." \textit{The Review of Economics and Statistics}, 90(2): 300-323.
	
	\bibitem[Autor, Levy and Murnane(2003)]{ALM2003} \textbf{Autor, David H., Frank Levy and Richard J. Murnane} (2003). ``The Skill Content of Recent Technological Change: An Empirical Exploration." \textit{The Quarterly Journal of Economics}, 118(4): 1279-1333.
	
	\bibitem[Baker, Bloom and Davis(2011)]{BBD2013} \textbf{Baker, Scott R., Nicholas Bloom and Steven J. Davis} (2013). ``Measuring economic policy uncertainty." \textit{Chicago Booth research paper}, 13-02.
	
%	\bibitem[Banister and Cook(2011)]{BC2011} \textbf{Banister, Judith and George Cook} (2011). ``China's Employment and Compensation Costs in Manufacturing through 2008." \textit{Monthly Labor Review}, 39-52.
	
	\bibitem[Blinder(2009)]{B2009} \textbf{Blinder, Alan S.} (2009). ``How Many US Jobs Might be Offshorable." \textit{World Economics}, 10: 41-78.
	
	\bibitem[Blinder and Krueger(2013)]{BK2013} \textbf{Blinder, Alan S. and Alan B. Krueger} (2013). ``Alternative Measures of Offshorability: A Survey Approach." \textit{Journal of Labor Economics}, 31(2): 97-128.
	
	\bibitem[Brynjolfsson and McAfee(2014)]{BM2014} \textbf{Brynjolfsson, Erik and Andrew McAfee} (2014). \textit{The Second Machine Age: Work, Progress, and Prosperity in a Time of Brilliant Technologies.} WW Norton \& Company.
	
		\bibitem[Brynjolfsson, McAfee and Spence(2014)]{BMS2014}\textbf{Brynjolfsson, Erik, Andrew McAfee and Michael Spence} (2014). ``New World Order. Labor, Capital, and Ideas in the New Power Law Economy." \textit{Foreign Affairs}, 93(4): 44-53.
	
	\bibitem[Ebenstein, Harrison, McMillan and Phillips(2014)] {EHMP2014} \textbf{Ebenstein, Avraham, Ann Harrison, Margret McMillan and Shannon Phillips} (2014). ``Estimating the Impact of Trade and Offshroing on American Workers Using the Current Population Surveys." \textit{Review of Economics and Statistics}, 96(4): 581-595. 
	
	\bibitem[Feenstra and Hanson(1999)]{FH1999} \textbf{Feenstra, Robert C. and Gordon H. Hanson} (1999). ``The Impact of Outsourcing and High-Technology Capital on Wages: Estimates for the United States, 1979-1990." \textit{Quarterly Journal of Economics}, 114(3): 907-940.

	\bibitem[Fierro(2013)]{F2013} \textbf{Fierro, Gabe} (2013). ``Extracting and Formatting Patent Data from USPTO XML." \textit{Fung Technical Report} No. 2013.06.16, University of California, Berkeley.
	
	\bibitem[Gentzkow and Shapiro(2010)]{GS2010} \textbf{Gentzkow, Matthew and Jesse M. Shapiro} (2010). ``What drives media slant? Evidence from US daily newspapers." \textit{Econometrica}, 78(1): 35-71.

	\bibitem[Goos and Manning(2007)]{GM2007} \textbf{Goos, Maarten and Alan Manning} (2007). ``Lousy and Lovely Jobs: The Rising Polarization of Work in Britain." \textit{Review of Economics and Statistics}, 89(1): 118-133.
	
	\bibitem[Goos, Manning and Salomons(2014)]{GMS2014} \textbf{Goos, Maarten, Alan Manning and Anna Salomons} (2014). ``Explaining Job Polarization: Routine-Biased Technological Change and Offshoring." \textit{American Economic Review}, 104(8): 2509-2526.
	
	\bibitem[Grossman and Rossi-Hansberg(2008)]{GR2008} \textbf{Grossman, Gene M. and Esteban Rossi-Hansberg} (2008). ``Trading Tasks: A Simple Theory of Offshoring". \textit{American Economic Review}, 98(5): 1978-1997.

	\bibitem[Hall, Jaffe and Trajtenberg(2001)]{HJT2001} \textbf{Hall, Bronwyn H., Adam B. Jaffe and Manuel Trajtenberg} (2001). ``The NBER Patent Citation Data File: Lessons, Insights and Methodologial Tools". \textit{NBER Working Paper No. 8498}.
	
	\bibitem[Hansen, McMaho and Prat(2014)]{HMP2014} \textbf{Hansen, Stephen, Michael McMahon and Andrea Prat} (2014). ``Transparency and Deliberation within the FOMC:
a Computational Linguistics Approach". CEPR Discussion Paper 9994.

	\bibitem[Hinze and Schmoch(2005)]{HS2005} \textbf{Hinze, Sybille and Ulrich Schmoch} (2005). ``Opening the Black Box." \textit{Handbook of Quantitative Science and Technology Research}, Springer Netherlands, 215-235.
	
	\bibitem[Hummels, Jørgensen, Munch and Xiang(2014)]{HJMX2014} \textbf{Hummels, David, Rasmus Jørgensen, Jakob Munch and Chong Xiang} (2014). ``The Wage Effects of Offshoring: Evidence from Danish Matched Worker-Firm Data." \textit{American Economic Review}, 104(6): 1597-1629.
	
	\bibitem[Lewis(2011)]{L2011} \textbf{Lewis, Ethan} (2011). ``Immigration, Skill Mix, and Capital Skill Complementarity." \textit{The Quarterly Journal of Economics}, 126(2): 1029-1069.
	
%	\bibitem[Marin(2014)]{M2014} \textbf{Marin, Dalia} (2014). ``Globalisation and the Rise of the Robots." \textit{VoxEU.org}, CEPR. \href{http://www.voxeu.org/article/globalisation-and-rise-robots}{\texttt{[link]}}
	
	\bibitem[Magerman, Van Looy and Song(2010)]{MVS2010} \textbf{Magerman, Tom, Bart Van Looy and Xiaoyan Song} (2010). ``Exploring the feasibility and accuracy of Latent Semantic Analysis based on text mining techniques to detect similarity between patent documents and scientific publication." \textit{Scientometrics}, 82(2): 289-306.	

	\bibitem[Manning, Raghavan and Schütze(2009)]{MRS2009} \textbf{Manning, Christopher D., Prabhakar Raghavan and Hinrich Schütze} (2009). \textit{An Introduction to Information Retrieval.}  Cambridge University Press.
	
	\bibitem[Schrader and Laaser(2009)]{SL2009} \textbf{Schrader, Klaus and Claus-Friedrich Laaser} (2009). ``Globalisierung in der Wirtschaftskrise: wie sicher sind die Jobs in Deutschland." \textit{Kieler Diskussionsbeiträge} No. 465
	
%	\bibitem[Timmer, Erumban, Los, Stehrer and de Vries(2014)]{TELSdV2014} \textbf{Timmer, Marcel P., Abdul Azeez Erumban, Bart Los, Robert Stehrer and Gaaitzen J. de Vries} (2014). ``Slicing Up Global Value Chains." \textit{Journal of Economic Perspectives}, 28(2): 99-118.
		
\end{thebibliography}

%\vspace*{0.7cm}
%\begin{appendix}
%\newpage
%\section{Figures}
%\newpage

%\end{appendix}



%%%%%%%%%%%%%%%%%%%%%%%%%%%%%%%%%%%%%%%%%%%%%%
\end{document}
