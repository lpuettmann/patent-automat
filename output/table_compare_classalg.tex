\begin{table}
\begin{small}
\begin{threeparttable}
\caption{{\normalsize Evaluation of different classification algorithms}}
\label{table:table_compare_classalg}
\begin{tabular}{rlllllllll}
\toprule 
 & \textbf{Algorithm1} & \textbf{automat} & \textbf{Bessen-Hunt} & \textbf{Always "No"} & \textbf{Always "Yes"}  \tabularnewline 
Accuracy & 0.80 & 0.78 & 0.77 & 0.73 & 0.27 & & & &  \tabularnewline 
Precision & 0.62 & 0.58 & 0.62 & NaN & 0.27 & & & &  \tabularnewline 
Recall & 0.67 & 0.57 & 0.32 & 0.00 & 1.00 & & & &  \tabularnewline 
F-measure & 0.64 & 0.57 & 0.42 & NaN & 0.42 & & & &  \tabularnewline 
AUC & 0.76 & 0.71 & 0.63 & 0.50 & 0.50 & & & &  \tabularnewline 
MCC & 0.51 & 0.42 & 0.32 & NaN & NaN & & & &  \tabularnewline 
Number "Yes" & 162 & 147 & 77 & 0 & 560 & & & &  \tabularnewline 
\bottomrule 
\end{tabular} 
\begin{tablenotes}
\small
\item\textit{Note:} "automat" classifies all patent as automation patent that include "automat" at least once.\item Bessen-Hunt: (anywhere in patent:) ``software" OR (``computer" AND ``program") ANDNOT ((in title:) ``chip" OR ``semiconductor" OR ``bus" OR ``circuit" OR ``circuitry") ANDNOT ((anwhere in patent:) ``antigen" OR ``antigenic" OR ``chromatography").
\item F-measure: balanced F-measure which is the evenly weightened harmonic mean between Precision and Recall. 
\item AUC: Area under (the receiver operating) curve.
\item MCC: Matthew's correlation coefficient.
\item\textit{Source:} USPTO, Google and own calculations.
\end{tablenotes}
\end{threeparttable}
\end{small}
\end{table}
